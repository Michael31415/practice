\documentclass[titlepage]{article}
\usepackage[ukrainian]{babel}
\usepackage{listings}
\usepackage{amsmath}
\usepackage{fontspec}
\setmainfont{Times New Roman}
\usepackage{setspace}
\spacing{1.7}
\usepackage{fancyhdr}
\usepackage{titling}
\usepackage[a4paper, top = 20mm, bottom = 20mm, left = 25mm, right = 15mm]{geometry}
\usepackage{graphics}

\lstset{
	basicstyle=\large,
	breaklines = true,
	language = matlab,
	breakatwhitespace = true,
	numbers = left,
	numberstyle = \tiny,
	columns = flexible,
	frame=tb,
	tabsize=4
}

\makeatletter
\renewcommand\normalsize{%
\@setfontsize\normalsize{14pt}{15pt}}
\makeatother

\edef\name{spl\textunderscore23.m}

\newcommand\makelisting[1]{\begingroup\spacing{1}\lstinputlisting[title=#1]{../#1} \vspace{1cm} \endgroup}
\newcommand\mysection[1]{\begingroup\center\section*{#1}\endgroup}
\newcommand\eq[1]{\begin{equation*}\begin{split}#1\end{split}\end{equation*}}
\newcommand\eqc[1]{\begin{equation*}\begin{cases}#1\end{cases}\end{equation*}}

\preauthor{\begin{flushright}}
\postauthor{\end{flushright}}

\fancypagestyle{empty}{%
	\renewcommand{\headrulewidth}{0pt}
	\cfoot{\normalsize2016}
	\chead{\uppercase{київський національний університет імені тараса шевченка}}
}

\begin{document}

\title{\normalsize Ебаная хуйня, блядь \protect\\ Побудова інтерполяційного сплайну}
\author{\vspace{5cm}Виконав: \protect\\ студент 4-го курсу \protect\\ спеціальність математика \protect\\ Сивак Назар}
\date{}
\maketitle
\mysection{Постановка задачі}
Побудувати інтерполяційний сплайн $S(x, u)$ другого степеня дефекту 1, з крайовими умовами типу ІІІ, використовуючи метод $2M$ для пошуку системи лінійних рівнянь та метод релаксації для її розв'язку.
\mysection{Теоретичні відомості}
Інтерполяційний сплайн другого степеня дефекту 1 --- це така функція $S(x, u) \in C^1([a,b])$, що для інтерполяційної сітки $X$ та функції $u$ виконується:
\eq{
\forall x \in [X_i, X_{i+1}): S(x, u)=a_2^i(x-X_i)^2 - a_1^i(x-X_i) + a_0^i; \forall i: S(X_i, u) = u(X_i).
}
Але для сплайнів парного степеня використовують іншу сплайнову сітку $\{x_i\} \subset [a,b]$, точки якої зазвичай кладуть посередині відрізків інтерполяційної сітки:
\eq{
x_i = (X_{i-1} + X_{i}) / 2,  i=\overline{2,n};\quad x_1 = X_1, x_{n+1} = X_{n+1}.
}
Тоді
\eq{
\forall x \in [x_i, x_{i+1}): S(x, u)=a_2^i(x-X_i)^2 - a_1^i(x-X_i) + a_0^i; \forall i: S(X_i, u) = u(X_i). 
}
Для побудови системи рівнянь для коефіціентів використаємо другі похідні $2a_i = M_i = S^{''}(X_i, u)$. З теорії відомі такі обмеження на $M_i$ (навчальний посібник "Сплайн-функції та її застосування"):
\eq{
&h_{i-1}M_{i-1} + 3(h_{i-1} + h_i)M_i + h_iM_i = 8u(X_{i-1};X_i;X_{i+1})(h_{i-1} + h_i);\\
&h_{i-1}M_{i-1} + 3(h_{i-1} + h_i)M_i + h_iM_i = 8(u(X_i; X_{i+1}) - u(X_{i-1}; X_i)).
}
Де $h_i = X_{i+1} - X_i$. Тоді отримуємо систему рівнянь:
\eqc{
&M_1 = A;\\
&h_{i-1}M_{i-1} + 3(h_{i-1} + h_i)M_i + h_iM_i = 8(u(X_i; X_{i+1}) - u(X_{i-1}; X_i)), i = \overline{2,n-1};\\
&M_n = B;
}
Для якої можна записати матрицю
\[\left(\begin{array}{ccccc|c}
1 & 0 & 0 & \ldots & 0 & A\\
h_1 & 3(h_1 + h_2) & h_2 & \ldots& 0  & 8(u(X_2; X_3) - u(X_1; X_2))\\
\ldots &\ldots &\ldots &\ldots &\ldots & \ldots\\
\ldots & 0 &h_{n-2} & 3(h_{n-2} + h_{n- 1}) & h_{n-1} & 8(u(X_{n-1}; X_n) - u(X_{n-2}; X_{n-1})\\
0 & \ldots & 0  & 0 & 1 & B
\end{array}\right)\]
В умові вимагається розв'язання системи лінійний рівнянь методом квадратного кореня, який в свою чергу вимагає ермітовості матриці, тому цю систему перетворюємо в симетричну відніманням від другого та передостаннього рядка відповідно першого та останнього, помножених на відповідні коефіціенти.

Для обрахування саме коефіціентів сплайну використовуються наступні формули:
\eqc{
&a_i = \frac{M_i}{2};\\
&c_i = u_i;\\
&b_1 = u(X_1, X_2) - \frac{1}{8}h_1(3M_1 + M_2);\\
&b_i = u(X_{i-1};X_i) + \frac{1}{8}h_1(M_{i-1} + 3M_i), i=\overline{2,n}.
}
\mysection{Практична реалізація}

Функція, що здійснює перевірку правильності побудови сплайну: побудову графіків та розрахунок сіткової норми.
\makelisting{perevirka.m}

Функція, що здійснює побудову сплайна.
\begingroup\spacing{1}\lstinputlisting[title=\name]{../spl_23.m} \vspace{1cm} \endgroup
\end{document}