\documentclass[titlepage]{article}
\usepackage[ukrainian]{babel}
\usepackage{listings}
\usepackage{amsmath}
\usepackage{fontspec}
\setmainfont{Times New Roman}
\usepackage{setspace}
\spacing{1.7}
\usepackage{fancyhdr}
\usepackage{titling}
\usepackage[a4paper, top = 20mm, bottom = 20mm, left = 25mm, right = 15mm]{geometry}
\usepackage{graphics}

\lstset{
	basicstyle=\large,
	breaklines = true,
	language = matlab,
	breakatwhitespace = true,
	numbers = left,
	numberstyle = \tiny,
	columns = flexible,
	frame=tb,
	tabsize=4
}

\makeatletter
\renewcommand\normalsize{%
\@setfontsize\normalsize{14pt}{15pt}}
\makeatother

\newcommand\makelisting[1]{\begingroup\spacing{1}\lstinputlisting[title=#1]{../#1} \vspace{1cm} \endgroup}
\newcommand\mysection[1]{\begingroup\center\section*{#1}\endgroup}

\preauthor{\begin{flushright}}
\postauthor{\end{flushright}}

\fancypagestyle{empty}{%
	\renewcommand{\headrulewidth}{0pt}
	\cfoot{\normalsize2016}
	\chead{\uppercase{київський національний університет імені тараса шевченка}}
}

\begin{document}

\title{\normalsize Учбова практика \protect\\ Побудова інтерполяційного сплайну}
\author{\vspace{5cm}Виконав: \protect\\ студент 4-го курсу \protect\\ спеціальність математика \protect\\ Шатохін Михайло}
\date{}
\maketitle
\mysection{Постановка задачі}
Необхідно побудувати інтерполяційний сплайн $S(x, u)$ другого степеня дефекту 1, з крайовими умовами типу II.
\mysection{Теоретичні відомості}
За умовою задачі необхідно побудувати інтерполяційний сплайн $S(x, u)$ другого степеня дефекту 1, тобто за даною сіткою $X=(a=x_1<x_2<\ldots<x_{n_1}=b)$ побудувати функцію з неперервною першою похідною таку, що 
\[\forall x \in [x_i, x_{i+1}): S(x, u)=a_2^i(x-x_i)^2 - a_1^i(x-x_i) + a_0^i; \forall i: S(x_i, u) = u(x_i)\]
Якщо сітка має $n+1$ точку, то попередні умови породжують обмеження на коефіцієнти, а саме: $2n$ обмежень випливає з необхідності рівності сплайну та функції у вузлах сітки (по два обмеження на кожний відрізок $[x_i, x_{i+1}]$) та $n-1$ обмеження через неперервність похідної (по одному в кожній внутрішній точці сітки). З двома крайовими умовами отримуємо $3n+1$ обмеження, але тільки $3n$ змінних, тому таку задачу неможливо розв'язати в загальному випадку. Скоротимо кількість крайових умов до однієї --- вимагатимемо тільки в лівому кінці проміжку інтерполювання:
\[S^{''}(x, u) = A;\]
Для побудови системи рівнянь для коефіціентів використаємо другі похідні $M_i = S^{''}(x_i+, u)$, де береться права границя через те, що друга похідна сплайна не обов'язково неперервна. Згадавши явний вигляд сплайна отримуємо $a_2^i = \frac{M_i}{2}$, але для скорочення коефіцієнтів покладемо $N_i = a_2^i = \frac{M_i}{2}$. Тепер запишемо умови в термінах нових змінних:
\begin{equation*}
\begin{split}
&1.S(x_i, u) = u(x_i) = u_i \implies a_0^i = u_i;\\
&\text{Покладемо  } x_{i+1} - x_i = h_i, \Delta_i = u_{i+1} - u_i;\\
&2.S(x_{i+1}-,u) = u_{i+1} \implies N_ih_i^2 + a_1^ih_i = u_{i+1} - u_i \implies a_1^i = \frac{\Delta_i}{h_i} - N_ih; \\
&3.S^{'}(x_{i+1}-, u) = S^{'}(x_i+, u) \implies 2N_ih_i + a_1^i = a_1^{i+1} \implies\\
& \frac{\Delta_i}{h_i} + N_ih_i = \frac{\Delta_{i+1}}{h_{i+1}} - N_{i+1}h_{i+1} \implies N_ih_i + N_{i+1}h_{i+1} =  \frac{\Delta_{i+1}}{h_{i+1}} - \frac{\Delta_i}{h_i};\\
&4.S^{''}(a, u) = A \implies 2N_1 = A.
\end{split}
\end{equation*}
Таким чином отримаємо систему лінійний рівнянь для $N_i$:
\[\left(\begin{array}{ccccc|c}
2 & 0 & 0 & \ldots & 0 & A\\
h_1 & h_2 & 0 & \ldots& 0  & \frac{\Delta_{2}}{h_{2}} - \frac{\Delta_1}{h_1}\\
0 & h_2 & h_3 & \ldots & 0 & \frac{\Delta_{3}}{h_{3}} - \frac{\Delta_2}{h_2}\\
\ldots &\ldots &\ldots &\ldots &\ldots & \ldots\\
0 & \ldots & 0  & h_{n-1} &h_n & \frac{\Delta_{n}}{h_{n}} - \frac{\Delta_{n-1}}{h_{n-1}}
\end{array}\right)\]
В умові вимагається розв'язання системи лінійний рівнянь методом квадратного кореня, який в свою чергу вимагає ермітовості матриці, тому цю систему перетворюємо в симетричну додаванням до кожного рядка наступного, помноженого на відповідний коефіціент.
\mysection{Практична реалізація}
В практичній реалізаціі з метою упорядкування коду створено декілька функцій та трохи змінено їх роль у программі. Обрахунок сплайну в точках сітки $T$ відбувається безпосередньо в головній (перевіряючій) частині та ця сітка не передається в функцію що будує сплайн. Дійсно, для побудови сплайну не має бути важливо в яких точках він буде потім обраховуватись. Ця функція, що могла би називатися $spl\_21$, в реалізаціі має назву $CreateSpline$, вона повертає коефіціенти побудованого сплайну (як матрицю $3xn$) та функцію, що обраховує сплайн та його похідні. 

Функція, що здійснює перевірку правильності побудови сплайну: побудову графіків та розрахунок сіткової норми.
\makelisting{main.m}
Функція, що здійснює побудову сплайна.
\makelisting{CreateSpline.m}
Побудова матриці за допомогою других похідних $M_i = 2N_i$.
\makelisting{CreateSEMatrix.m}
Розв'язання системи лінійних рівнянь за допомогою методу квадратного кореня.
\makelisting{SolveSE.m}
Формування коефіцієнтів сплайну.
\makelisting{FormSpline.m}
\end{document}